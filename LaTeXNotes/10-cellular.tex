\section{Cellular Security}

\subsection{1G --- Analog}

\paragraph{Overview}
Introduced in the early 1980s to connect to the telephone network
(\textit{Public Switched Telephone Network PSTN}). Medium access control: split
bandwidth with FDMA, with one call using the same frequency in both directions.
Suppprts handover between different base stations.

\paragraph{No security}
Identification via serial and phone numbers. Control messages as analogue
tones.

\underline{Problems:} eavesdropping (privacy), mobile cloning (billing fraud).

\subsection{2G --- GSM}

\paragraph{Overview}
Introduced in the early 1990s. Digital voice and control messages, enabling
features like: compression, error correction, less power, SMS, security
mechanisms. We focus on the \textit{Global System for Mobile Communications
	GSM}.

\paragraph{Architecture}
See \autoref{fig:2g-arch}.

Medium access control: \hl{FDMA with distinct uplink/downlink frequency channels}.
TDMA\footnote{Time-division multiple access} to support 8 speech channels on
the same frequency.

Different channels for traffic and control frames (e.g.\ paging channel, random
access channel, access grant channel).

\begin{figure}[h]
	\centering
	\includegraphics[scale=0.45]{images/10-2g-arch.png}
	\caption{Architecture of 2G}%
	\label{fig:2g-arch}
\end{figure}

\begin{figure}[h]
	\centering
	\includegraphics[scale=0.4]{images/10-2g-channels.png}
	\caption{Setup of incoming call (BTS ``pages'' the MS) over the different channels}%
	\label{fig:2g-channels}
\end{figure}

\paragraph{Security model}
The \hl{main goal was to prevent the misuse of a subscriber identity by a third
	party}. Everything based on symmetric shared keys $K_i$. The key is stored in
the \textit{Home Location Register HLR} of the provider and on the
\textit{subscriber identification module SIM} card (and never leaves it).

\underline{Algorithms:}
A3 for authentication, A5 for encryption, A8 for key derivation.
Initially secret, but A5 leaked in the mid 90s, and got reverse engineered in 1999.

\paragraph{GSM authentication}
See \autoref{fig:2g-authentication}. From the shared key $K_i$ and a random
challenge a session key $K_c$ is derived. Together with the frame nonce/counter
$F_n$ it is used to encrypt the plaintext frame $m_i$.

Note that there is no mutual authentication (\hl{only the phone is authenticated})
and messages can be replayed! Also note that since A3 and A8 are executed on
the SIM card, the operator can choose these!

\begin{figure}[h]
	\centering
	\includegraphics[scale=0.5]{images/10-2g-authentication.png}
	\caption{2G (Authentication) Flow}
	\label{fig:2g-authentication}
\end{figure}

\paragraph{GSM encryption}
Goal: fast in hardware. Two variants A5/1 (strong) and A5/2 (weak, not
discussed here).

\underline{A5/1:} stream cipher with a \textit{Linear Shift Feedback Register LSFR} and 64 bit security.
Registers are initialized with the key $K_c$ and the frame counter $F_n$ to create the keystream, which is then XORed with the plaintext.
See \href{https://web.archive.org/web/20120326211404/http:/l-system.net.pl/crypto/A5\_1\_stream\_cipher.svg}{here} for a visualization of the LSFR.

\underline{Attack approach:} \\
Known plaintext/ciphertext pair
$\overset{XOR}{\longrightarrow}$ keystream
$\longrightarrow$ secret internal state
$\overset{solve\; LSE\; with\; 64\; eqns}{\longrightarrow}$ key

\begin{figure}[h]
	\centering
	\includegraphics[scale=0.5]{images/10-2g-a5.png}
	\caption{2G A5 Linear Shift Feedback Registers}
	\label{fig:2g-a5}
\end{figure}

\paragraph{A5/1 attacks}
Attacks of A5/1 evolved over time (2000--2010). The first were not very
practical (requiring many known plaintexts or special-purpose hardware). Types
included: Time-Memory Tradeoff Attack (Biryukov 2000, Nohl 2010), Correlation
Attack (Brakan \& Birham 2005), Guess and Determine Attack (Gendrullis et al.
2008), Fast Near Collisions (Zhang 2019). \\ Note that a few known plaintexts
are reasonable (known control frames).

\href{https://media.blackhat.com/bh-us-10/whitepapers/Nohl/BlackHat-USA-2010-Nohl-Attacking.Phone.Privacy-wp.pdf}{\underline{Karsten Nohl (2010):}}
2TB precomputed table (mapping from keys/state to keystreams, 1 month computation with 4 GPUs), 64 bit plaintext, 5 sec attack time (lookup).%
\footnote{Nohl omitted some attack details, later provided by Lu (2015).}
\\
Challenge: reducing table size ($2^{64}$ unreasonable).

% \paragraph{Time-Memory Tradeoff Attack} (Hellman 1980)
% \begin{enumerate}
% 	\item Precompute chains $x_1 \overset{f}{\rightarrow} x_2 \overset{f}{\rightarrow} x_3 \overset{f}{\rightarrow} ...$ where $f$ is A5/1.
% 	Store start point SP and end point EP.
% 	\item Attack: Create chain for the observed keystream.
% 	Check if any element in the chain matches a known EP.
% 	Re-create chain from SP to find the likely key (the element before the keystream in the re-computed chain).
% \end{enumerate}
% Tradeoff: longer chains mean less storage but more computation during the attack.

% \begin{figure}[h]
% 	\centering
% 	\includegraphics[scale=0.3]{images/10-2g-tmto.png}
% 	\caption{Time-Memory Tradeoff Attack}
% 	\label{fig:2g-tmto}
% \end{figure}

\paragraph{Rainbow tables} (Oechslin 2003)
\\
Solves issue of collisions in chains (leading to reduced keyspace coverage as chains merge).
Uses different variant $f_i$ (``color'') for each chain link.
Different lookup details, but same idea.

\paragraph{A5/1 attacks summary}
Previous ideas generally applicable to stream ciphers, not just to A5/1. Main
enabler is the short key size (64 bit). Nevertheless it lasted quite well
considering when it was designed and under which hardware constraints, and
given that there is still ongoing research.

\paragraph{A8 attacks}
Setup: $K_c$ known, want to recover $K_i$. \\ 1998: COMP128 hash inverted in
hours (effectively only 54 bit)\\ 2002: Faster recovery using side-channels. \\
Mitigation: Operators replace A8 (OTA update, new SIM).

\paragraph{GSM --- no integrity protection}
No integrity protection defined, due to too much overhead (voice frame has 144
bits). Also special use case: dropping frames/retransmission is undesired, and
small voice frame modifications are acceptable.

\paragraph{GSM --- no mutual authentication}
Recall that the phone does not authenticate the base station. Probable
reasoning (1980s): expensive equipment, call encrypted anyway. But: commercial
fake BS (2000s), USRP (2010), etc enable user identification + tracking and
MITM.

\subsection{3G --- UMTS}

\paragraph{Overview}
\textit{Universal Mobile Telecommunication System UMTS} introduced in the early 2000s.

Radio link uses \hl{wideband code-division multiple access W-CDMA},
separate per-user spreading codes, distinct uplink and downlink frequency
bands.

\paragraph{Protocol}
New \textbf{authentication and key agreement (AKA)} protocol. Provides \hl{mutual
	authentication, mutual replay protection, integrity protection}. Also used in
4G+5G (more or less).

Similar design principles like GSM: operator and SIM trusted, phone and visited
networks untrusted, minimize communication with home network.

\underline{Remarks:} (see \autoref{fig:3g-aka-details})
\begin{enumerate}
	\item Both the SIM card and the operator maintain the sequence number SQN (against
	      replays)
	\item The IMSI\footnote{International Mobile Subscriber Identity} is send before the
	      authentication (enabling tracking, see later)
	\item Function $f_1, f_2, f_3, f_4, f_5$ are operator-specific
	\item Loose synchronisation required
	\item Integrity key IK for integrity protection
\end{enumerate}

Further details on the authentication, encryption and integrity
protection\footnote{Mandatory for signalling + control messages, optional for
	data. Based on a 8-round Feistel network to be fast in hardware.} functions can
be found in the slides but are omitted in this summary.%

\begin{figure}[h]
	\centering
	\includegraphics[scale=0.4]{images/10-3g-aka-overview.png}
	\caption{AKA High Level Flow}
	\label{fig:3g-aka-overview}
\end{figure}

\begin{figure}[h]
	\centering
	\includegraphics[scale=0.7]{images/10-3g-aka-details.png}
	\caption{AKA Detailed Flow}
	\label{fig:3g-aka-details}
\end{figure}

\paragraph{Cryptography Summary}
Authentication and Key Agreement protocol formally verified with respect to
authentication and confidentiality (2001). Two known but impractical attacks on
encryption (interesting for research though). TLDR: Good for now.

\paragraph{Denial of Service}
Commercial jammers available for a few hundred dollars (though use is
illegal!). \\ Approaches:
\begin{itemize}
	\item Insert noise on physical layer.
	\item Jam/block paging messages (control layer). Difficult, requires synchronisation
	      with the victim.
	\item Answer paging messages faster than the victim, causing the victim's reply to be
	      ignored. Possible because (a) paging occurs before authentication and (b) base
	      stations cover large areas.
\end{itemize}

\paragraph{MITM/Fake BS}
Though AKA authentication is mutual, a \hl{MITM perform a downgrade attack to force
	the phone to use GSM} (due to co-existence). Also, in 3G the MiTM can learn the
IMSI at the start of AKA.

\underline{Practical considerations:}
Which frequency to use --- allocated or unallocated?
What cell id to use --- a new, unknown one?
Jam legitimate BS to get victims to connect to yours?
\\
$\implies$ Though setting up a fake BS is easy, detecting it is easy as well.

\paragraph{Femtocells}
Operator provides a ``mini-BS'' to customers to improve local (indoor)
coverage. The femtocell box relays from the radio link to the operator network.

Vulnerable because they are easier to access then normal base stations (high up
on a tower). Gaining access gives a perfect MITM position, having the keys for
the gateway to the operator as well as for the radio link.

\paragraph{User tracking}
\hl{Identity (IMSI) is sent before authentication}. Even though a temporary identity
(TMSI) is issued, the IMSI is reused on occasions. Thus, user tracking is
possible to some extent, but not addressed by the spec (tradeoff possibility of
abuse versus increased complexity).

\underline{Approaches for identity protection}:
\begin{itemize}
	\item \textit{Pseudonyms}: send pseudonym when starting AKA, with the home network always returning a new pseudonym
	      (encrypted\footnote{Note that this encryption can be done symmetrically with the shared key, since the home network could use the pseudonym to look it up.}, so that the serving network cannot read it).\\
	      Challenge: requires synchronisation, and thus a recovery process.
	      However, it is hard to design a recovery process that cannot be abused to learn the IMSI.
	\item \textit{Public key encryption}: store home network public key on SIM, encrypt IMSI.
	      Defined as optional in 5G.
	      Pro: no state that needs to be synchronised.
	      Con: asymmetric cryptography is expensive.
\end{itemize}

\subsection{Signalling System 7 --- SS7}

Signalling network used in GSM + 3G to route calls, coordinate roaming, deliver
SMS, etc. Defined in the 80s/90s.

Initially only a few participating, mutually trusted operators. But: Soon grew
to 1000+ operators and third-party service providers, and SS7 access could be
purchased at a low price.\\ $\implies$ Trust assumption violated.

Open source software and specs online, anybody with network access can send SS7
commands with a Linux computer.\\ $\implies$ Assumption on expensive equipment
violated.

Attacks in 2014 by Engel (see the
\href{https://media.ccc.de/v/31c3\_-\_6249\_-\_en\_-\_saal\_1\_-\_201412271715\_-\_ss7\_locate\_track\_manipulate\_-\_tobias\_engel}{31C3
	talk here}), discussed below.

\begin{figure}[h]
	\centering
	\includegraphics[scale=0.5]{images/10-ss7.png}
	\caption{SS7 Architecture}
	\label{fig:ss7}
\end{figure}

\paragraph{Location tracking}
\hl{Phone locations are stored in the \textit{Gateway Mobile Location Center GMLC}},
access to which requires authentication (e.g. law enforcement). However, by
requesting the routing info from the HLR and with that the cell id from the
\textit{Mobile Switching Center MSC} where the user is currently logged in, one
can work around this to still get a rough location.

\begin{figure}[h]
	\centering
	\includegraphics[scale=0.4]{images/10-ss7-location.png}
	\caption{SS7 Location Tracking}
	\label{fig:ss7-location}
\end{figure}

\paragraph{Intercepting Calls}
See \autoref{fig:ss7-calls}.
\begin{enumerate}
	\item Attacker overrides the \textbf{GSM Service Control Function} \textit{(gsmSCF)} in the
	      MSC with their own.
	\item When the target makes a call, the MSC now contacts the attacker.
	\item The attacker learns the phone number and rewrites it towards their recording
	      proxy.
	\item MSC sets up call to the proxy.
	\item Proxy bridges call to the intended receiver.
\end{enumerate}

\begin{figure}[h]
	\centering
	\includegraphics[scale=0.4]{images/10-ss7-calls.png}
	\caption{SS7 Intercepting Calls}
	\label{fig:ss7-calls}
\end{figure}

\paragraph{SS7 Summary}
\hl{Legacy system with outdated trust model}. Bad network management (open
interfaces, no authentication or access control to control messages). Attacks
are independent of cryptography and the radio link (i.e. work from far away).
\\ Some issues were fixed, and LTE has a new signalling system (Diameter).

\subsection{4G  --- LTE}

\paragraph{Overview}
\textit{Long-Term Evolution LTE} introduced in 2008.
\\
\underline{Updated architecture:}
fully packet switched, new core network (\textit{Evolved Packet Core EPC}, fully packet-switched), new radio network (\textit{Evolved UMTS Terrestrial Radio Access Network E-UTRAN}), but interoperable with legacy systems.
\\
\underline{Updated physical layer:}
\textit{Orthogonal Frequency Division Multiplexing OFDM} (downlink with orthogonal sub-carriers, single-carrier uplink), multiple antennas (MIMO).

\paragraph{Architecture and Terminology}
See \autoref{fig:4g-arch}.
\begin{itemize}
	\item \textit{User Equipment UE} (MS): the mobile handset
	\item \textit{Evolved Node B eNB} (BS): the base station
	\item \textit{Mobility Management Entity MME}: handles signalling via the \textit{Non-access stratum NAS}, UE authorisation, S-GW selection
	\item \textit{Home Subscriber Server HSS} (HLR): subscriber database, user authentication
	\item \textit{Serving Gateway S-GW}:  routes user data packets
	\item \textit{Packet Gateway P-GW}: connects to external network, routing, filtering
\end{itemize}

\begin{figure}[h]
	\centering
	\includegraphics[scale=0.5]{images/10-4g-arch.png}
	\caption{LTE Architecture}%
	\label{fig:4g-arch}
\end{figure}

\paragraph{Network Protocol Stack}
See \autoref{fig:4g-network-stack}. From top to bottom:
\begin{itemize}
	\item \textit{Non-access stratum NAS}: mobility management, tracking area update, etc
	\item \textit{Radio Resource Control RRC}: AKA, paging messages, system information broadcast, etc.
	\item \textit{Packet Data Convergence Protocol PDCP}: compression, optionally encryption+integrity
	\item \textit{Radio Link Control RLC}: error correction, segmentation, frame ordering
	\item \textit{MAC layer}: manages access to radio link
\end{itemize}
Note that everything below the PDCP layer is unencrypted.
Thus most sniffing+spoofing attacks focus on the layers below.

\begin{figure}[h]
	\centering
	\includegraphics[scale=0.5]{images/10-4g-network-stack.png}
	\caption{LTE Network Protocol Stack}%
	\label{fig:4g-network-stack}
\end{figure}

Uplink congestion control

\paragraph{Security Overview}
Authentication: similar to AKA.

\begin{figure}[h]
	\centering
	\includegraphics[scale=0.3]{images/10-4g-aka.png}
	\caption{LTE Architecture}%
	\label{fig:4g-aka}
\end{figure}

Encryption/integrity: 3 variants EEA1, EEA2, EEA3 and EIA1, EIA2, EIA3.

Other: extended key hierarchy, option for longer keys (256 bit), handover
between eNBs (X2), backhaul (S1) protection.

\textbf{Authentication} is \hl{required in both} control and user plane.

\textbf{Encryption} is \hl{optional for both} the user and control plane,
but often used.

\textbf{Integrity} protection is \hl{mandatory for the control plane}, but
not for user data

\paragraph{Key hierarchy}
to limit attack possibilities and impact. Master key $K$ (128 bits, stored on
HSS+SIM), confidentiality key $CK$, integrity key $IK$, etc.
\begin{figure}
	\centering
	\includegraphics[scale=0.5]{images/10-4g-key-hierarchy.png}
	\caption{LTE Key Hierarchy}
	\label{fig:4g-key-hierarchy}
\end{figure}

\paragraph{Backhaul + EPC protection}
For backhaul, the LTE spec recommends physical protection. For EPC, the spec is
vague (``division of security domains''). In practice both are secured using
standard IP security practices (VPN, PKI).

\paragraph{Handover + Key Separation}
Reduced attack surface and key scope by limiting key lifetime of $K_{eNB}$.
E.g. different keys for different eNBs/cells.

\paragraph{Location tracking} \mbox{} \\
\underline{Background:}
The service area is divided intro \textit{tracking areas TAs} containing multiple cells (each controlled by an eNodeB that broadcasts information such as the TA code, mobile network code, cell ID).
UE sends IMSI with the Attach request, in which the operator assigns temporary identifiers that are used subsequently (TMSI, GUTI\footnote{Global unique temporary identifier}).

\underline{Adversary:}
Goal: learn user locations.
Capabilities: transmit/receiver radio signals, possible with commercial USRPs.
Advantage: GUTI re-allocation depends on operator, possibly not changed for multiple day.

\underline{Attack:}
\begin{enumerate}
	\item Set up fake BS.
	\item Monitor user presence in TA.
	\item Learn precise location: actively send unprotected \textit{RRC Connection
		      Reconfig} messages, to which the UE responds with a \textit{Measurement Report}
	      containing the signal strengths of neighbouring cells and its GPS location.
\end{enumerate}

\underline{Analysis:}
Not all signalling/control messages are integrity protected/authenticated.
Spec allows this explicitly for troubleshooting (availability versus privacy).

\paragraph{MITM} \mbox{} \\
\underline{Background:}
MAC layer assigns unique \textit{Radio Network Temporary Identifiers RNTIs} to distinguish UEs.
eNodeB uses \textit{Downlink Control Information DCI} to notify UEs when radio resources are available.
Also recall that EEA2 uses AES-CTR for encryption (XORs keystream with plaintext).

\underline{Attack:}
\begin{enumerate}
	\item Identify UE from encrypted traffic: observe connection establishment, learn
	      TMSI+RNTI, use paging to map TMSI to phone number.
	\item Modify/redirect encrypted traffic: often uplink is encrypted but not integrity
	      protected. Xor ciphertext with ``manipulation mask'' (try-and-error).
\end{enumerate}

\underline{Analysis:}
Identifiers on lower layers, encryption on higher layers.
Integrity protection optional.

\paragraph{Jamming/DoS}
Brute-force jamming always possible, but requires a lot of power. Instead,
targeting specific control channels can be effective, too (see next point).

% \paragraph{Signal Overshadowing (SigOver)} \mbox{} \\
% \underline{Idea:}
% Broadcast signals are not integrity protected (e.g. \textit{System Information Blocks SIBs)}.
% Spoof them by overshadowing specific frames of the legitimate broadcasts (providing a misconfiguration to prevent the UE from connecting), for example by sending a ``Configuration (SIBs) Cell Barred: True".

% \underline{Analysis:}
% Low jamming-to-signal ratio (J/S), thus stealthy (not as obvious as a fake BS).
% Only downlink affected, thus undetected by the base station.
% Challenges: time+frequency synchronisation with the legitimate signal, distance/delay estimation to the UE, phone may quickly reconnect to another cell.

% \underline{Commercial Mobile Alert Service CMAS} messages (``presidential alerts'') are also delivered via SIB12, allowing signal overshadowing.

\paragraph{Keystream Reuse Attack / ReVoLTE} \mbox{} \\
\underline{Idea:} IV for EEA comprised of a counter, radio bearer ID and radio direction.
\\
Unfortunately, many operators re-use bearer IDs and reset counter for subsequent calls (exactly what we need!).
Adversary can initiate a second call just after the target call and record both calls, both unencrypted and ciphered.
\\
Fix: don't repeat id and counter.

\paragraph{Signal Overshadowing (SigOver)} \mbox{} \\
\underline{Idea:}
SigOver is a signal injection attack that exploits the fundamental weakness of physical layer in Long-Term Evolution (LTE). Since LTE communication is based on an open medium, a legitimate signal can potentially be counterfeited by a malicious signal. In addition, although most LTE signaling messages are protected from modification using cryptographic primitives, broadcast messages in LTE have never been integrity protected.\footnote{\href{https://github.com/SysSec-KAIST/sigover_injector}{SigOver}}

This attack has several advantages and differences when compared with existing
attacks using a fake base station. For example, with a 3 dB power difference
from a legitimate signal, the SigOver demonstrated a 98\% success rate when
compared with the 80\% success rate of attacks achieved using a fake base
station, even with a 35 dB power difference. Given that the SigOver is a novel
primitive attack, it yields five new attack scenarios and implications.

\underline{Analysis:}
Low jamming-to-signal ratio (J/S), thus stealthy (not as obvious as a fake BS).
Only downlink affected, thus undetected by the base station.
Challenges: time+frequency synchronisation with the legitimate signal, distance/delay estimation to the UE, phone may quickly reconnect to another cell.


\paragraph{Adaptive Overshadowing (AdaptOver)} \mbox{} \\
\underline{Idea:}
Overshadowing the legitimate response to a NAS Service Request issuing a NAS Service Reject, which results in the UE not trying to connect again for $12-20$ hours.

\underline{Analysis:}
Low jamming-to-signal ratio (J/S), thus stealthy (not as obvious as a fake BS).
Only downlink affected, thus undetected by the base station.

\paragraph{LTrack --- IMSI Extractor} \mbox{} \\
\underline{Idea:}
Overshadowing the legitimate response to a NAS Service Request issuing a NAS Identity Request, and sniffing the uplink, which results in the attacker learning the IMSI of the victim UE.

\paragraph{LTrack --- Passive variant} \mbox{} \\
This attack allows a fully passive localization of the victim.
On the Downlink, Base Station notifies UE about the propagation delay between them, specifies a ring around a base station and travels unencrypted on MAC layer.
In LTE-A, UE connects to multiple base stations.
On the Uplink, Reference Signals used for channel correction. Propagation delay is channel condition. Observe propagation delay from multiple points. More accurate than 2G and 3G since 4G requires tighter synchronization.
\begin{figure}[h]
	\centering
	\includegraphics[scale=0.135]{images/10-ltrack.png}
	\caption{LTrack Passive Localization}%
	\label{fig:ltrack}
\end{figure}

\paragraph{4G Summary}
New crypto algorithms, new core network. Small security improvements (key
hierarchy, handover protection), but not yet perfect. \\ Types of attacks:
SigOver, fake base stations, man-in-the-middle. \\ Attack properties:
stealthiness/detectability, power requirement, J/S ratio.

\subsection{5G}

\paragraph{Overview}
Currently being deployed (2019/2020). \\ \underline{Radio link}: \textit{5G New
	Radio NR}, optimised OFDM, massive MIMO, two frequency ranges (FR1: sub-6GHz,
FR2: mmWave range, 24-100GHz, high-throughput, high-bandwidth). Beam management
to steer beams with a phase array allows connecting more devices. \\
\textit{Time Division Duplex TDD} allows the same channel/frequency to be used
for both up- and downlink, with different time intervals for different
directions.\footnote{Compare this with LTE which used FDD: the uplink and
	downlink used different frequencies.} On one hand this allows flexible
allocation, but on the other it requires precise synchronisation!%

\paragraph{Attacks}
Some ideas as research is ongoing.
\begin{itemize}
	\item Beam stealing: attack beam training to steer beams away from victims (shown for
	      IEEE 802.11ad)
	\item Broadband jamming (DoS): increasingly difficult due to large bandwidth (power
	      constraint) $\implies$ need protocol-aware spoofing for DoS (challenge: tight
	      synchronisation).
	\item PSS\footnote{Primary Synchronisation Signal} spoofing: soft takeover:
	      synchronise to cell, introduce PSS at correct timing then slowly move peak away
	      (see GNSS \autoref{sec:gps-spoof}).
\end{itemize}
\begin{figure}[h]
	\centering
	\includegraphics[scale=0.3]{images/10-5g-suci.png}
	\caption{SUCI Catcher}%
	\label{fig:suci}
\end{figure}

\paragraph{5G Security Summary}
Similar crypto algorithms. Better replay protection for AKA (SIM generates
nonces). User tracking mitigations (SIM can encrypt IMSI/TMSI with home
operator's public key, stricter policies for changing temporary ids).

\begin{figure}[h]
	\centering
	\includegraphics[scale=0.5]{images/10-overview.png}
	\caption{Cellular Security Summary}%
	\label{fig:overview}
\end{figure}

\subsection{Questions}

\textbf{What capabilities does an adversary need to perform location tracking on a 4G network?} He needs to get a hold of the IMSI of the UE. This can be done overshadowing a legitimate message after a service or attach request with a Identity Request message, to which the UE will happily reply with its IMSI. Or, the attacker could use two passive base stations to record the propagation delay and use triangulation to calculate the origin of the signal.

\textbf{Describe the security requirements (Authentication, Encryption and Integrity) for the control and user plane in 4G. What does the standard prescribe/recommend? Do the operators follow those recommendations?} MISSING DESCRIPTION. Authentication is required in both the control and user plane, encryption is optional for both but often used, integrity is mandatory for control plane but not user data.