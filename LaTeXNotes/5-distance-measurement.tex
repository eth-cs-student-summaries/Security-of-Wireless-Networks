\section{Secure Ranging}

Secure Ranging, or distance measurement, is commonly used, for example, in wireless car keys, contact tracing in a pandemic, payments, automation.

\textbf{Replay attacks} are an issue, allowing an attacker to \hl{make devices appear physically closer} (e.g.\ if the device naively use the observed signal strength to derive the distance).
\\
\underline{Goals:}
(Provably) secure ranging, protecting against all logical and physical attacks and all attacker abilities, with a \hl{focus on preventing distance reduction}, that should bind distance to identity.

\paragraph{Current techniques (overview)} \mbox{} \\
Non-Time-of-Flight:
\begin{itemize}
	\item Received Signal Strength Indication [RSSI] (WiFi, Bluetooth, 802.15.4, NFC, RFID) --- \textit{insecure}
	\item (Multi-carrier) phase measurement\footnote{The distance is proportional to the phase.}
	      --- \textit{insecure}
	\item Frequency-Modulated Continuous-Wave FMCW --- \textit{insecure}
\end{itemize}

Time-of-Flight:%
\footnote{
	Calculate via $d = c \cdot (t_{tof} - t_{proc}) / 2$
	where $t_{tof}$ is the time between sending and receiving the signal and $t_{proc}$ is the known processing time on the responding device. \\
	In general, manipulating time is harder than manipulating signal properties (strength, phase).}
\begin{itemize}
	\item Chirp Spread Spectrum (802.15.4 CSS) --- \textit{insecure}
	\item Ultra Wide Band UWB (802.15.4z) --- \hl{\textit{proposed}}
	\item WiFi 802.11az, 5G --- \textit{first efforts to secure OFDM-based}
\end{itemize}

\paragraph{Model}
On a logical level, we have a \textbf{verifier V} and a \textbf{prover P}, between which we want to measure the distance.
A \textbf{malicious party M} attacks this.

% See also the \href{https://www.iacr.org/workshops/fse2013/slides/Slides02.pdf}{Brands-Chaum protocol} (not discussed in HS20).

Additionally, we assume the worst case for the users but the best case for the attacker (bad channel/noise/multipath versus perfect channel $\rightarrow$ attacker guesses will look like noise).

\begin{figure}[h]
	\centering
	\includegraphics[scale=0.4]{images/5-model.png}
	\caption{Model of the distance bounding scenario}%
	\label{fig:ranging-model}
\end{figure}

\paragraph{Types of attacks (frauds)}
\begin{itemize}
	\item \textbf{Distance fraud:} A dishonest $P$ tries to change its distance to $V$.
	\item \textbf{Mafia fraud:} Honest $V, P$ being attacked by an external $M$.
	\item \textbf{Terrorist fraud:} Dishonest $P$ and $M$ collude to change $P$'s distance.
	\item \textbf{Distance hijacking:} Dishonest $P$ leverages an honest $P$ to change its distance.
\end{itemize}


\paragraph{Logical Layer}
Distance reduction prevented using challenge-response protocols with time measurements (i.e., distance-bounding). V and P share a secret key unknown to M. V and P exchange messages that are (in part) unpredictable to M. STS cannot be predicted by M (before it is sent).

\begin{figure}[h]
	\centering
	\includegraphics[scale=0.8]{images/5-sts.png}
	\caption{Scrambled Timestamp Sequence}%
	\label{fig:sts}
\end{figure}

\paragraph{Distance Commitment}
Distance commitment is the Time of Arrival measured over a public preamble and then verified using payload pulses generated using UWB pulse reordering. The timing of the preamble is binding. An attacker needs to advance payload if he advances the preamble. It's the way of knowing the distance between transmitter and receiver.
As the attacker needs to advance the payload as well, he needs to guess and send before the legitimate pulse arrives, which gives him immediate feedback, and the attacker can then adjust the subsequent pulses in order to reach the needed net energy level.


\paragraph{Distance Bounding}
A prover who wants to prove its position should quickly receive $N_V$, compute $Ff(N_V,N_P)$ and send it.
The verifier estimates prover's processing time $=t_p$. \hl{If the attacker's processing time is $0$, then he can cheat by $t_p/2$}. Thus, ideally, $t_p=0s$; usually, though, $t_p=1-2ns$ which means that the attacker can only shorten the distance by $15-30cm$.
\begin{figure}[h]
	\centering
	\includegraphics[scale=0.4]{images/5-dist-bounding.png}
	\caption{Distance Bounding}%
	\label{fig:dist_bounding}
\end{figure}
An evolution of this is Challenge reflection with Channel Selection, where the prover only needs to reflect $N_V$ and the verifier does all the work. This reduces $t_p$ to $0$.

\paragraph{Physical Layer: Representing bits as pulses (UWB)}
There are two design options to represent a bit: either with a \textbf{single strong pulse} or a \textbf{sequence of weaker pulses}.
Single pulses may not be detected reliably (distance, interference), so the aggregate over several pulses is the preferred representation.

\begin{figure}[h]
	\centering
	\includegraphics[scale=0.3]{images/5-single-multi-pulse.png}
	\caption{Bit representation: single- versus multi-pulse}%
	\label{fig:single-multi-pulse}
\end{figure}


\subsection{Early-Detect/Late-Commit attack (ED/LC)}
This attacks shortens the distance, since the receiver receives the first symbols earlier than it should have.

\begin{enumerate}
	\item Attacker sends noise (at time $T_A$)
	\item Attacker learns correct symbol (at time $T_{ed}$)
	\item Attacker commits to correct symbol (at time $T_{lc}$), by sending the remaining pulses such that the sum over all pulses matches.
\end{enumerate}

Note that this attack is not possible with single pulses.
A single pulse is usually $1-2$ ns long, so the attacker can cheat by at most $15-30$ cm (performance/security tradeoff).

\begin{figure}[h]
	\centering
	\includegraphics[scale=0.3]{images/5-edlc.png}
	\caption{Early-Detect/Late-Commit attack (ED/LC)}%
	\label{fig:edlc}
\end{figure}

\paragraph{ED/LC Solution 1: Pulse Reordering UWB-PR}
Interleave pulses of subsequent symbols according to some cryptographic reordering.
Thus the start and end time of a symbol is unpredictable, and the attacker can only guess.
\\
The probability of an attack's success decreases with the increase of the number of reordered bits.

\begin{figure}[h]
	\centering
	\includegraphics[scale=0.45]{images/5-pulse-reordering.png}
	\caption{Pulse Reordering}%
	\label{fig:pulse-reordering}
\end{figure}

\paragraph{ED/LC Solution 2: Variance Based Detection}
Statistically analyze the received versus the expected pulses.
This forces the attacker to ``guess better'' to reduce the variance and make their error indistinguishable from the noise.

\begin{figure}[h]
	\centering
	\includegraphics[scale=0.4]{images/5-variance-based.png}
	\caption{Variance Based Detection}%
	\label{fig:variance-based}
\end{figure}

\paragraph{ED/LC Solution 3: Scrambled Timestamp Sequence}

After the preamble (high correlation) which is used for ToF \textit{estimation},
send a Scrambled Timestamp Sequence STS (encrypted with a shared secret, low autocorrelation) to use for ToF \textit{verification}.
\\
See IEEE 802.15.4z.
Security not formally proven and unclear!

\begin{figure}[h]
	\centering
	\includegraphics[scale=0.35]{images/5-scrambled-timestamp.png}
	\caption{Scrambled Timestamp Sequence}%
	\label{fig:scrambled-timestamp}
\end{figure}

\subsection{IEEE 802.15.4z LRP versus HRP}
\begin{table}[h]
	\centering
	\begin{tabular}{ll}
		\hline
		Low Rate Pulse LRP                & High Rate Pulse HRP                      \\ \hline
		Can use single pulse              & No single pulse (energy too low)         \\
		Multi-pulse with UWB-PR efficient & UWB-PR + variance-based seem inefficient \\
		Open  and simple security specs   & No open security analysis                \\
		Low-cost, low-energy              & More transparency needed                 \\ \hline
	\end{tabular}
\end{table}


In HRP, receivers cannot decode individual pulses. Instead, they first need to find the maximum autocorrelation peak (which is not the correct ToA) and then look for the Early Path (leading edge/correct ToA)
\begin{figure}[h]
	\centering
	\includegraphics[scale=0.3]{images/5-early-peak.png}
	\caption{Early Peak}%
	\label{fig:earlypeak}
\end{figure}

\subsection{Ghost-Peak Attack}
This attack shows that physical-layer attacks on HRP are a real threat. Apple/NXP/Qorvo use hidden ToA estimation algorithms making it hard to evaluate if their HRP ranging systems are truly secure. Instead, these algorithms should be public and comprehensively evaluated.
Like with ciphers/signatures/MACs, their security should rely only on the unpredictability of the STS, not on the secrecy of the algorithms.
It resulted in a successful reduction of the perceived distance of $4-12$ meters.

\begin{figure}[h]
	\centering
	\includegraphics[scale=0.45]{images/5-ghost-peak.png}
	\caption{Ghost Peak Attack}%
	\label{fig:ghostpeak}
\end{figure}

\paragraph{Main idea}
The attack basically consists in injecting a packet in-sync (micro-seconds) with the legitimate transmission but with varying power of Preamble, SFD, STS.\@ The attack is based on the assumptions that there is no knowledge of the keys shared between victim devices and the attacker is not able to predict the Scrambled Timing Sequence (STS).
The necessary equipment is inexpensive and easily purchasable.

\paragraph{Message Time of Arrival Code MTAC}
New class of cryptographic primitives that verify the integrity of message arrival time.
E.g.\ single-pulse, UWB-PR, Variance-based detection.

\begin{figure}[h]
	\centering
	\includegraphics[scale=0.3]{images/5-mtac.png}
	\caption{Message Time of Arrival Code MTAC}%
	\label{fig:mtac}
\end{figure}

\paragraph{Verifiable Multilateration}
Multiple verifiers with known locations want to determine the position of a prover.
C.f. GPS trilateration.
E.g.\ to position autonomous cars using cell towers.


\paragraph{Future Work}
\begin{itemize}
	\item Secure Positioning (e.g.\ verifiable multilateration)
	\item WiFi 802.11 and 5G ranging (some initial work)
	\item Efficient implementation + deployment
\end{itemize}
