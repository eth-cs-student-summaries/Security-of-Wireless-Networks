\section{Security of Global Navigation Satellite Systems --- GNSS}

\paragraph{Overview}
Orbiting \hl{satellites transmit} their \hl{location} and a \hl{precise timestamp}.
Receivers collect these navigation messages and their arrival time and use
\textbf{trilateration} to calculate their own position. \footnote{There are of
	course issues with special and general relativity that mess with the time.}
Satellites are positioned in such a way that \hl{at least four of them are
	always in sight} from any point on Earth. If more satellites are visible, a
more precise localization is possible.

\textbf{Remark:} even though space is three-dimensional, \hl{we need at least 4 satellites} (thus 4 equations) because we also need to \hl{solve for the clock-error}, which is otherwise unknown.

Three segments: users, satellites, ground control.

\paragraph{Types of start at the receiver}
\underline{intuition:} \hl{the more the receiver knows, the faster it can lock on} to the satellites in view and get a new localization.
\begin{itemize}
	\item \textbf{Cold start:} receiver knows nothing.
	\item \textbf{Warm start:} receiver remembers last calculated position, almanac and UTC time.
	\item \textbf{Hot start:} receiver remembers last calculated position, almanac and UTC time and the last satellites in view.
	\item \textbf{+ Assisted GPS:} receiver gets helped by the cellular network downloading ephemeris and estimating its position using triangulation from cell towers.
\end{itemize}

\paragraph{Signalling}
Each satellite \textbf{modulates the navigation message} with a \textbf{spreading code} (coarse acquisition C/A for civilians (public), precision P/Y for military (secret)).
The spread signal is then modulated onto a carrier. \hl{Individual satellites use
	individual spreading codes to allow distinction}. For the civilian GPS,
spreading codes are \textbf{public}.

GPS sends on two carrier frequencies at the same time, L1 (1575.42 MHz = 10.23
MHz $\times$ 154) and L2 (1227.60 MHz = 10.23 MHz $\times$ 120).\footnote{This
	only applies to military. The civilian C/A is only transmitted on L1.} Apart
from jamming resistance and redundancy, this also allows to calculate the
ionospheric delay error. \\ Due to atmospheric attenuation, down on Earth the
GPS signal is well below the thermal noise.%

\paragraph{Navigation message}
Each message consists of 25 frames. Each frame takes 30 sec to transmit, so the
total time, in the case of a cold start, is 12.5 min. \\ Each frame contains:
satellite clock + health data, 2x ephemeris (orbit details), other data +
almanac (orbital + clock details). Navigation messages are transmitted by the
satellite at a rate of $50$ bits per second.

\paragraph{Time of Arrival TOA}
The time of arrival is the travel time of the signal from the satellite to the
receiver and it's \hl{used to calculate the distance and, thus, the receiver
	position}. It's found by sliding the spreading code over the received message
until a correlation peak is found.

\paragraph{Spoofing attacks}
\hl{Messages are unauthenticated} (for practical reasons, or otherwise they would
become too long).

By sending stronger signals, which overshadow the legitimate
ones, an attacker can modify the \textit{navigation message contents}
(transmission time, satellite location) or their \textit{time of arrival}
(retransmitting captured signals with a temporal shift), resulting in a wrong
location being calculated.

This is an issue in civilian GPS (messages can be
generated and delayed) as well as in military GPS (messages can only be delayed
since they are encrypted). Unfortunately, commercial GPS signal generators are
becoming increasingly cheap.

\paragraph{Impct of attacks}
\hl{Jamming} can cause DoS, which is bad but it's possible to recover in other ways.
However, it's \hl{useful to assist the attacker in other attacks} (for example jamming the
legitimate signal before injecting a rogue signal). Obtaining a rogue position
is bad for smart/autonomous vehicles, high-value asset tracking, especially if
you don't detect it and keep trusting the wrong position.
\subsection{Spoofing Detection and Mitigation}\label{sec:gps-spoof}

\paragraph{Types of countermeasures}
\begin{itemize}
	\item \textbf{Infrastructure/protocol:} for example, cryptographic \hl{authentication} of navigation messages
	\item \textbf{Receivers:} for example, using \hl{physical-layer characteristics of the signal} to validate the signal as well as the calculated position/velocity/time (e.g.\ direction of arrival, carrier phase, signal strength, etc\dots)
\end{itemize}

\paragraph{Angle of Arrival --- AoA}
\hl{Use multiple antennas} (e.g.\ on both ends of a ship) \hl{to calculate the angle of
	arrival through the phase difference} and the known distance between the
antennas (see beam steering, \autoref{fig:beam-steering}). In a spoofed
scenario, the angles would all be very similar. Restricts the locations from
which the attacker can successfully spoof.

\underline{Problems:}
Attacker can use drones to spoof signal from more realistic angle.
Reflection of legitimate signal of buildings (thus reaching the receiver at a shallower angle) could be wrongly classified as spoofing.
Computationally expensive phase measurement.
Hardware modification.

\paragraph{Monitor Signal Characteristic Changes}
Over time, \hl{monitor signal properties such as AGC} (Automatic Gain Control),
noise level, number of satellites, spatial diversity (AoA) or the
autocorrelation peak. Abrupt changes in any of these indicate presence of
spoofing.

\paragraph{Seamless takeover attack}
The attacker starts transmitting a copy of a legitimate GPS signal in sync with
the original one, but at low power, having no influence on the receiver. Then
the attacker slowly starts increasing the power, until the receiver prefers the
attacker signal. Now the attacker can change the GPS signal, and the receiver
will keep following.

\begin{figure}[h]
	\centering
	\includegraphics[scale=0.3]{images/4-seamless-takeover.png}
	\caption{Seamless takeover attack}%
	\label{fig:seamless-takeover}
\end{figure}

\paragraph{SPoofing REsistance GPS rEceiver SPREE}
Leverage peak tracking (of \underline{all} signal peaks) to detect seamless
takeover attacks. Navigation message inspection detects content spoofing.

\paragraph{Cryptographic approach (Kuhn)}
\begin{enumerate}
	\item At time $t$: satellite uses secret spreading code. Receiver uses a broadband
	      receiver to capture the entire band.
	\item At time $t+dt$: Satellite disclosed code, signing the disclosure with its
	      secret key. Receiver verifies signature, de-spreads the signal.
\end{enumerate}

\underline{Advantages:}
Prevents fake signal generation and individual signal delay.

\underline{Disadvantages:}
Requires pre-shared public satellite keys.
Slightly inefficient (longer latency until signal lock).
Requires loose synchronization (after $dt$ the attacker knows the spreading code and can spoof).
Does NOT prevent full-band delay.
Relay/replay attacks (record signal and replay full band in another location in real time).

\paragraph{Galileo OSNMA}
Galileo Open Service Navigation Message authentication offers a way of
\hl{authenticating messages} that's sustainable for the GNSS infrastructure.

Solutions that use public key cryptography are not suitable, since the
signatures would be too long to fit in the limited bandwidth available, and are
computationally too expensive. Symmetric cryptography is not suitable as well,
since it relies on shared secrets that would need to be stored in every
receiver, which one can't trust to actually \underline{stay} secret.

OSNMA uses \hl{time-delayed authentication}. The satellite generates a random
value $K_n$ and hashes it $n$ times and obtains a chain. Obviously, the hash
function is irreversible, so you can't compute $K_n$ if you have $K_{n-1}$. The
satellite sends the digital signature and then starts sending the messages in
the form of \[ M_1, K_0,MAC(M_1, K_1)\rightarrow \text{ then, after some time, }
	\rightarrow M_2, K_1,MAC(M_2, K_2).\]
Basically, it \hl{discloses the key only after that the message encrypted with that key
	has been received} --- and thus the key, which is valid only once, is no longer
useful to any attacker. The receiver verifies the first key against the
signature using the public key, then verifies the key against the previous key
by hashing it and then verifies the previous message using the key disclosed at
that moment.

\paragraph{Security analysis}
It's important to note that this solution is \hl{still susceptible to anticipation
	attacks}, which are harder but possible and use techniques like Early Detect-Late Commit and Forward Error Estimation, and also delaying attacks, butthese can be detected with a clock offset test.
